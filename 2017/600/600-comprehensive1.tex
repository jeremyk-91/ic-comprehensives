\documentclass[12pt, a4paper]{article}
\usepackage{fullpage}
\usepackage{graphicx}
\usepackage{amsmath}
\usepackage{amssymb}
\usepackage[hidelinks]{hyperref}
\usepackage{listings}
\usepackage{epigraph}
\renewcommand{\epigraphsize}{\footnotesize}
\setlength{\epigraphwidth}{.5\textwidth}
\lstset{
  language = C,
  basicstyle = \ttfamily \small,
  numbers = left,
  numberstyle = \footnotesize,
  showstringspaces = false
}
\usepackage{times}

\title{600 Comprehensive I}
\author{Jeremy Kong}
\date{2 January 2017}

\begin{document}
\maketitle

\noindent A couple of quick notes:
\begin{itemize}
\item I'd intend for the timelimit for this paper to be 3 hours and 20 minutes (200 minutes total).
\item This question was constructed by a student at Imperial College London, and is not intended to be representative of actual examination questions for any specific course at Imperial.
\item The material students have learned from the compulsory modules in first and second year should be sufficient to answer all questions.
\item Each question is worth 20 marks; the maximum mark is thus 100.
\item The questions are roughly arranged in what I think is difficulty order. Also, \textit{within each question} the difficulty generally increases as you progress from one part to the next.
\end{itemize}

\newpage
$\:$
\newpage 

\section{A Matter of Perspectives}
\epigraph{One of these things is not like the others, \\
Which one is different -- do you know? \\
Can you tell which thing is not like the others, \\
I'll tell you if it is so...
}{\textit{Big Bird, Sesame Street}}

\noindent In this question, you will be given various sets of items. For each item in the set, identify how it might be ``not like the others". For example:
$$
2, 8, 9, 76
$$

\begin{itemize}
\item $2$ is not like the others, because it is a prime number.
\item $8$ is not like the others, because it is a perfect cube.
\item $9$ is not like the others, because it is a perfect square.
\item $76$ is not like the others, because it is a multiple of $19$.
\end{itemize}

\noindent Obvious hacks that are not in the spirit of the question (e.g. ``this is the only one shorter than $N$ characters", ``this is the only one that contains precisely two instances of the letter V") will not be accepted. Also, answers that explicitly reference other members of the set are not allowed. \\[12pt]
\noindent Have fun!

\begin{enumerate}
\item 
\begin{enumerate}
\item Kruskal's algorithm
\item Ford-Fulkerson algorithm
\item Prim's algorithm
\item Edmonds-Karp algorithm
\end{enumerate}
% Kruskal's is the only one that is typically implemented with a disjoint set union data structure
% Ford-Fulkerson is the only one that might not run in time polynomial in the input
% Prim's is the only one that works independently on connected components of a graph
% Edmonds-Karp is the only one that is p-time and solves maximum flow
[4 marks]

\item
\begin{enumerate}
\item $\mathcal{N}(2, 1)$
\item $Geometric(0.5)$
\item $Binomial(10, 0.2)$
\item $Exponential(0.5)$
\end{enumerate}
% Normal is the only one that can go negative.
% Geometric is the only discrete probability distribution here that isn't bounded.
% Binomial is the only one that is bounded by 10.
% Exponential is the only one that has a mode of 0.
[4 marks]

\item
For this one, you're not allowed to use examples of words that are only accepted by each of the languages below.
\begin{enumerate}
\item $0^n 1^n 2^n$
\item $0^n 1^n$
\item $0^n 0^n$
\item $0^n$
\end{enumerate}
% 0^n 1^n 2^n is the only one that's context sensitive
% 0^n 1^n is a context-free but not regular language
% 0^n 0^n = 00^* is a regular language, but is not star-free
% 0^* is a star-free language (because you can use complement)
[4 marks]

\newpage

\item
\begin{enumerate}
\item \texttt{synchronized}
\item \texttt{Semaphore}
\item \texttt{ReentrantLock}
\end{enumerate}
% synchronized is the only one for which you can't guarantee fairness
% Semaphore is the only one that's not re-entrant
% ReentrantLock is the only one for which you can query the current owner of a lock, or synchronize sections which can't be expressed using block structure in a re-entrant way
[3 marks]

\item
\begin{enumerate}
\item JK flip flop
\item SR flip flop
\end{enumerate}
% JK flip flop allows you to use '11' to toggle the output, for SR that is undefined.
[1 mark -- in a set of 2 you can always take complements]

\item
\textbf{For this part, identify \textit{two} distinct ways in which each of these functions is not like the others. Also remember the point on obvious hacks.} \\[12pt] 
All code blocks below are in C++. The standard library function \texttt{pow(x, y)} takes in $x$ and $y$ as doubles, and returns $x^y$ as a double.
\begin{enumerate}
\item $\:$
\begin{lstlisting} 
double f1(int x) {
  return (x == 0) ? 1 : 2 * f1(--x);
}
\end{lstlisting}
\item $\:$
\begin{lstlisting} 
double f2(int x) {
  return pow(2, x);
}
\end{lstlisting}
\item $\:$
\begin{lstlisting} 
double f3(int x) {
  if (x == 0) {
    return 1;
  } else if (x < 0) {
    return 1 / f3(-x);
  }
  double y = f3(x / 2);
  return (x % 2 == 0) ? y * y : y * y * 2;
}
\end{lstlisting}
\item $\:$
\begin{lstlisting} 
double f4(int x) {
  return (x == 0) ? 1 :
    f4(x / 2) * f4(x / 2) * ((x % 2 == 0) ? 1 : 2);
}
\end{lstlisting}
\end{enumerate}
% f1 is the only one that will (likely) throw a stack overflow if you feed it -1, or the only one that runs in linear time (for a correct answer).
% f2 is the only one that's totally correct (!). Or the only one that runs in constant time.
% f3 is the only one that's correct... apart from the minimum value of an integer, where it will stack overflow. Or it's the only one that can stack overflow a machine with unbounded stack.
% f4 is the only one that is not partially correct, and it actually returns 2^|x|.
[4 marks]


\end{enumerate}

\newpage

\section{Monad}
\epigraph{Hello from the other side, \\
I must have called a thousand times \\
To tell you, I'm sorry for everything that I've done \\
But when I call, you never seem to be home...
}{\textit{``Hello", Adele}}

\noindent In this question, we investigate two flavours of \textit{monads} within computer science.

\begin{enumerate}
\item In functional programming, a \textit{monad} is used to encapsulate values of a certain data type. You are probably familiar with the \texttt{Maybe} monad in Haskell, which is used to allow a null value for data, much like Java or Guava's \texttt{Optional} class:

\begin{lstlisting}
data Maybe t = Just t | Nothing
\end{lstlisting}

Monads typically feature two functions:
\begin{itemize}
\item A \textit{unit function} (Haskell \texttt{return}), which constructs a monadic type from a given base type, and
\item A \textit{binding operation} (Haskell \texttt{>>=}) of type \texttt{(M t) -> (t -> M u) -> (M u)}, which in sequence retrieves the objects of type \texttt{t} from the first argument, maps the second argument (which is a function from \texttt{t} to \texttt{M u}) over the objects, extracts the objects of type \texttt{u} from the output of that, and finally combines them to a single \texttt{M u} object.
\end{itemize}

For example, consider the \texttt{Maybe} monad defined above. We can then define the operations as follows:

\lstset{
  language = Haskell,
  basicstyle = \ttfamily \small,
  numbers = left,
  numberstyle = \footnotesize,
  showstringspaces = false
}
\begin{lstlisting}
return :: a -> Maybe a
return x = Just x

(>>=) :: Maybe a -> (a -> Maybe b) -> Maybe b
m >>= g =
  case m of
    Nothing -> Nothing
    Just x -> g x        
\end{lstlisting}
\lstset{
  language = C,
  basicstyle = \ttfamily \small,
  numbers = left,
  numberstyle = \footnotesize,
  showstringspaces = false
}

\begin{enumerate}
\item The \texttt{List} type is also a monad. Construct the unit and binding functions for \texttt{List}, bearing in mind the semantics indicated above. %3 marks - let's give a few easy points

\item Haskell generally features \textit{lazy evaluation} and \textit{pure functions}. Explain briefly what each of these terms means, and discuss why they might not always be suitable in practical applications. %4 marks

\item Monads are used to handle I/O in Haskell as well. The I/O action \texttt{getLine} has type \texttt{IO String}; it is not a function (because it is clearly nondeterministic). It reads in a line from the console. Conversely, the I/O action \texttt{putStrLn} has type \texttt{String -> IO ()} and prints the input string to the console. \\[12pt]
Demonstrate how one can implement a Haskell program that reads in a line of input and echoes it to the console, removing all whitespace characters. \\[12pt]
(Hint: Recall that \texttt{fmap}, for functor \texttt{m}, transforms a function of type \texttt{a -> b} to one of type \texttt{m a -> m b}.) % 3 marks

\item The function \texttt{liftM2} has type \texttt{(a -> b -> c) -> m a -> m b -> m c}. It works as follows:
\begin{itemize}
\item The first argument is a function from the raw types \texttt{a} and \texttt{b} to \texttt{c}.
\item The second argument has values of type \texttt{a} encapsulated in the monad \texttt{m}; the third argument has values of \texttt{b} encapsulated in \texttt{m}.
\item The function should return the result of applying the function to the values encapsulated in the monads.
\end{itemize}
For example, \texttt{liftM2 (+) [1, 2] [4, 8] = [5, 9, 6, 10]}. Implement \texttt{liftM2} using \texttt{>>=}, briefly explaining your answer. %2 marks

\end{enumerate}

[12 marks]

\item \textit{Monadic second-order logic} (MSO) is an extension of first-order logic. First-order logic allows us to quantify over objects in a model; monadic second-order logic allows us to quantify over sets of objects -- for example, we could say that for every set $P$ and for every element $x$, $x$ is either in $P$ or not in $P$ (and not both in and not-in $P$). This would be written as 
$$
\forall P. \forall x.  (x \in P \vee x \notin P) \wedge \neg (x \in P \wedge x \notin P)
$$
(Full second-order logic also allows us to quantify over relations and functions.)

\begin{enumerate}
\item Translate to monadic second order logic: \textit{Any set that is bounded has a least upper bound.} You may assume the existence of an ordering $\leq$ on the objects. %2 marks

\item Consider a finite state machine $M$ with states $W$ and a left-total transition relation $R$ (that is, every state has at least one outgoing transition). 

\begin{enumerate}
\item Let $S$ be a sequence of states in $M$. Further, let $\sharp S$ be the number of elements in $S$. Translate from monadic second order logic to English:

$$
\exists X. \left( S[0] \in X \wedge S[\sharp S - 1] \notin X \wedge \bigwedge_{i=1}^{\sharp S - 1} S[i - 1] \in X \leftrightarrow S[i] \notin X \right)
$$

%2 marks
\item Using only the relation $R$ and \textit{as few additional predicates as required}, translate this statement from English to monadic second order logic.
\begin{quote}
\textit{
It is possible to partition $W$ into three sets, such that there exists a Hamiltonian path in $W$ whereby one never transitions from a node in a set to another node in the same set.
}
\end{quote}
You may use the number of states $|W|$ and/or the sequence-construct $S$ from the previous part.
%4 marks
\end{enumerate}

\end{enumerate}

[8 marks]

\end{enumerate}

\section{Potato Party}
\epigraph{A ``potato" flew around my room before you came \\
Excuse the mess it made, it usually doesn't rain
}{a variant of \textit{``Thinkin' About You", Frank Ocean}}

\noindent You are probably familiar with the notion of a \textit{hash function}. Let $h : \Sigma \rightarrow 2^N$ be a hash function, where $N$ is a positive integer and $\Sigma$ is a \textit{universe} of values of unbounded size. Of course, $h$ is not bijective -- we refer to the case where $i, j \in \Sigma, i \neq j$ and $h(i) = h(j)$ as a \textit{hash collision}. 
\begin{enumerate}
\item Hash functions are often used in cryptography to store message digests. Let $h$ be a hash function, and suppose that $h$ runs in time polynomial in $N$. Then, 
\begin{itemize}
\item $h$ is \textit{preimage-resistant} if given $h(i)$, it is \textit{difficult to compute} $i$.
\item $h$ is \textit{collision-resistant} if given $i$, it is \textit{difficult to compute} a $j \neq i$ such that $h(i) = h(j)$.
\item $h$ is \textit{second preimage-resistant} if given $h(i)$, it is \textit{difficult to compute} a $j \neq i$ such that $h(i) = h(j)$.
\end{itemize}
By \textit{difficult to compute}, we expect the computation to not be achievable (in the average case) in time polynomial in $N$.
\begin{enumerate}
\item Consider the following property: $h$ is \textit{strongly second preimage-resistant} if given $h(i)$, it is \textit{impossible to, in the best case, compute in polynomial time} a $j \neq i$ such that $h(i) = h(j)$. Explain why no hash function can be strongly second preimage-resistant. %2 marks

\item Prove that if $h$ is collision resistant, then it must also be second preimage-resistant. %2 marks
\end{enumerate}

[4 marks]

\item \texttt{etc/passwd} is a text file on Linux systems, which stores information about local user credentials. The file is named as such because it was previously used to store users' password data; however, typically in modern implementations the information used to validate user passwords isn't present any more. Records include usernames, user identifier numbers and group identifier numbers.
All users are allowed to read \texttt{etc/passwd} -- and this was true even when password information was stored in the file. 
\begin{enumerate}
\item Suggest why read access to \texttt{etc/passwd} might \textit{not} have been restricted only to the root user. (Hint: What information does the file have?)

\item One possible scheme is to compute a hash of user passwords and store them; when users try to verify their password, we compute its hash and check it against the stored hash. We can avoid hash collisions by choosing good hash functions that produce a reasonably long hash (say $512$ bits).
\\[12pt]
However, this is problematic because users with the same password can discover each others' passwords. We typically use what is called a \textit{salt}; for \textit{each} user $i$, in addition to their password $P_i$ we also generate and store a salt $S_i$, and store $h(P_i S_i)$.
\\[12pt]
\textit{Beyond} preventing users from discovering each others' passwords if they are exactly the same, outline reasons why salts are advantageous in terms of security.
\end{enumerate}

[6 marks]

\item Although there aren't necessarily security implications, we often still want hash functions to distribute keys across the range of possible hash values, for example if using hash-based data structures. A Java \texttt{HashSet} degenerates into a linked list if all of our objects hash to the same value.
\begin{enumerate}
\item Explain why Java's \texttt{String::hashCode} method is \textit{not} a hash function according to our definition. %2 marks

\item Consider the following block of Java code, which claims to remove all objects equal to a given object in a set of sets.

\lstset{
  language = Java,
  basicstyle = \ttfamily \footnotesize,
  numbers = left,
  numberstyle = \scriptsize,
  showstringspaces = false
}
\begin{lstlisting}
public static <E> void probeElement(Set<Set<E>> sets, E element){
  ExecutorService executor = 
          Executors.getFixedThreadPool(sets.size());
  sets.forEach(set -> executor.submit(new DeleteTask<>(set, element)));
  executor.shutdown();
  try {
    executor.awaitTermination(5, TimeUnit.MINUTES);
  } catch (InterruptedException e) {
    Thread.currentThread().interrupt();
  }
}

private static class DeleteTask<V> implements Runnable {
  private final Set<V> set;
  private final V elem;
  
  public DeleteTask(Set<V> set, V elem) {
    this.set = set;
    this.elem = elem;
  }

  @Override
  public void run() { set.remove(elem); }
}
\end{lstlisting}
\lstset{
  language = C,
  basicstyle = \ttfamily \small,
  numbers = left,
  numberstyle = \footnotesize,
  showstringspaces = false
}

The following comments are made on the code above in a pull request, with line numbers, where applicable, indicated in brackets. For each comment, indicate whether you agree with the PR comment or not, explaining your answer.
\begin{enumerate}
\item \hspace{0pt} [3] The perf of this implementation doesn't look good to me.
\item \hspace{0pt} [9] Could we add a line of logging here please?
\item \hspace{0pt} [23] This line may throw an exception which we aren't dealing with. Maybe this is OK, but you should know about this.
\item \hspace{0pt} This method will not behave as expected, especially if it's a HashSet.
\end{enumerate}
%8 marks
\end{enumerate}
[10 marks]

\end{enumerate}

\section{Part-Time Jobs}
\epigraph{Make the money, don't let the money make you \\
Change the game, don't let the game change you \\
I'll forever remain faithful \\
All my people stay true, I say --
}{\textit{``Make the Money", Macklemore}}

\noindent Many university students in the UK have been working part-time jobs; 2015 statistics from Endsleigh \cite{endsleigh} suggest that 77\% of students work part-time to fund their studies. The author did too; as an overseas student he got pretty close to the 20-hour weekly limit allowed by his student visa on several occasions. In addition to wages, additional benefits the author foresaw included adding further entries to his CV, gaining work experience and developing other skills as well. It was around early third year when the author learned about something called the \textit{time value of money} as well... \\[12pt]
Nonetheless, it's important to ensure that one's studies didn't suffer too much. The author fared pretty well in this regard, though he understands the concerns about this (and would not necessarily encourage everyone to take one up!)

\begin{enumerate}
\item Travis worked a part-time job as an undergraduate teaching assistant. He taught several first-year students various formal proof techniques. For example, let's consider this extract of x86-64 assembly code.

\begin{lstlisting}
// PRECONDITIONS:
// r8 has a reference to a sorted array of +ve 64 bit integers
// r9 has a reference to a sorted array of +ve 64 bit integers
// Both arrays are terminated by -1

mov r10, 0
mov r11, 0
mov ecx, 0
loop:
mov eax, [r8 + 8 * r10]
mov ebx, [r9 + 8 * r11]
cmp eax, -1
jeq done
cmp ebx, -1
jeq done
cmp eax, ebx
jne advance
inc ecx
inc r10
inc r11
jmp loop
advance:
jl less
inc r11
jmp loop
less:
inc r10
jmp loop
done:
// POSTCONDITION: ???

\end{lstlisting}

\begin{enumerate}
\item Formulate a postcondition for the above x86-64 assembly code snippet. Your postcondition should describe the value in the register \texttt{ecx} in relation to the arrays. Do not worry about the other registers.
\item Formally prove that your postcondition is established by the code.
\end{enumerate}
[7 marks]

\item Russell worked a part-time job at a quantitative finance company. There, he learned about various high-frequency trading (HFT) strategies. Clearly, instructions to buy or sell securities are going to involve numbers. However, it is possible for bits to be flipped in transmission. Supposing a 64-bit two's complement representation of numbers, an order to buy $100$ units of a stock could turn into $2^{62} + 100$ if a certain bit is flipped. \\[12pt]
A common risk mitigation strategy to the above can be resource-intensive, especially if said strategy may require network or computational overhead close to the size of the orders themselves! Given that we have a large volume of instructions to execute per round trip, but even if one or two bits are flipped we still want (most of) the rest of the transactions to go through, outline a suitable strategy to maintain safety and (to some extent) performance whilst minimising overhead. %Idea is something along the lines of a Merkle tree.
\\[12pt]
[5 marks]

\item Joel worked a part-time job as a software engineer. The spring term has $11$ weeks, and he would work one day every week. With probability $p$, the work day would be uneventful and he worked a number of hours distributed according to $U(8, 13)$; with probability $1 - p$, a high-priority ticket would arrive and he worked a number of hours distributed according to $U(12, 15)$. $p$ itself is distributed according to $U(a, b)$ with $0 \leq a \leq b \leq 1$. The number of hours Joel worked each week can be assumed conditionally independent given $p$. Let $T$ be the total number of hours Joel worked during the spring term.
\\[12pt]
Compute $E(T)$ and $Var(T)$ (this will be in terms of $a$ and $b$). Show all working.
\\[12pt]
You may find it useful to recall the following:
\begin{itemize}
\item We say that two random variables $X$ and $Y$ are conditionally independent given $Q = q$ iff $f_{X, Y | Q}(x, y | q) = f_{X, Y}(x, y)$ for all values of $x$ and $y$.
\item The mean of $U(a, b)$ is given by $\dfrac{a+b}{2}$; the variance is given by $\dfrac{(b-a)^2}{12}$.
\item The law of total variance: where $X$ and $Y$ are random variables,
$$
Var(X) = E(Var(X | Y)) + Var(E(X | Y))
$$
\end{itemize}
[8 marks]
\end{enumerate}

\newpage
\section{Last Shot}
\epigraph{It's been a long day, without you my friend \\
And I'll tell you all about it, when I see you again.
}{\textit{``See You Again", Wiz Khalifa ft. Charlie Puth}}
\begin{center}
\textbf{WARNING!} \\[11pt]
As the final question, this is designed to be the most difficult question in the paper. Good luck!
\end{center}
This question concerns the \textit{actor model} of concurrency and frameworks
built around this model. An \textit{actor} is treated as a primitive of concurrent
computation; actors communicate only by passing messages with one another.
Actors are also allowed to create additional actors, and internally maintain
local state.

\begin{enumerate}
\item The \textit{$\pi$-calculus} is a process calculus that had its design 
influenced by the actor model. We provide an outline of the syntax of the 
monadic $\pi$-calculus.

\begin{itemize}
\item $0$ means a process does nothing (or it can be used to terminate processes).
\item $x(y).P$ means that a process accepts an input on channel $x$, then binds
$y$ to instances of $x$, and then runs $P$.
\item $\overline{x} \langle y \rangle. P$ means that a process sends an output
$y$ on the channel $x$, and then runs $P$.
\item $P | Q$ means that $P$ and $Q$ are run in parallel.
\item $(\nu x) P$ means that a new channel $x$ is created, and then $P$ is run.
\item $!P$ means that infinitely many copies of $P$ are generated.
\end{itemize}

In terms of semantics, the main reduction rule is that of communication: the
process $\overline{x} \langle y \rangle. P | x(z). Q$ reduces to $P|Q[y/z]$,
meaning that $y$ replaces the free occurrences of $z$ in $Q$.

\begin{enumerate}
\item The process $\textbf{D} k$ is called a \textit{duplicator}; it receives
an input on channel $k$ and then generates two of that output on $k$ (once only). 
Express a duplicator in the $\pi$-calculus.
%2 marks. D k = k(x).(overline k<x>.0 | overline k<x>.0)

\item Let's suppose that we allow numeric inputs, and we want to design an
\textit{incrementor}; it receives a value on a channel and outputs a value that
is one higher (infinitely many times). A possible expression of this is
$!(x(v).\overline{x} \langle v + 1 \rangle)$.

The above solution is vulnerable to a race condition. Demonstrate a safety
violation (that is, a process submitting $v$ and \textit{not} receiving $v+1$) 
with the above expression of the incrementor.
%2 marks, the problem is that x is common.

\item Consider the \textit{dyadic $\pi$-calculus}, where in addition to the
above syntax we also support two new operations:

\begin{itemize}
\item $x(y, z).P$ means that a process accepts a vector of \textit{two} inputs on 
channel
$x$, binds the first input to instances of $y$ and the second to instances of $z$,
and then runs $P$.
\item $\overline{x} \langle y, z \rangle. P$ means that a process sends a vector of 
the outputs
$y$ and $z$ on the channel $x$, and then runs $P$.
\end{itemize}

The corresponding reduction rule would be that
$\overline{x} \langle y_1, y_2 \rangle. P | x(z_1, z_2). Q$ reduces to 
$P|Q[y_1, y_2/z_1, z_2]$.

The dyadic $\pi$-calculus is actually not more expressive than the monadic
$\pi$-calculus. This can be shown by providing an 
\textit{encoding} of the dyadic $\pi$-calculus into the monadic $\pi$-calculus;
specifically, we rewrite the two dyadic operators in the monadic $\pi$-calculus.

\begin{enumerate}
\item This is one possible encoding: $[[x(y, z).P]] = x(y).x(z).P$ and 
$[[\overline{x} \langle y, z \rangle. P]] = \overline{x} \langle y \rangle. \overline{x} \langle z \rangle. P$. However, this encoding is
incorrect -- provide a safety violation demonstrating this.
\item Propose a correct encoding of the dyadic $\pi$-calculus into the monadic
$\pi$-calculus. (Hint: The $\nu$ operator is the only construct that can generate
a name known to be unique.)
\item The \textit{polyadic} $\pi$-calculus is a further extension of the dyadic
$\pi$-calculus, allowing arbitrarily many inputs and outputs. Based on your result 
in the previous part, write down an encoding of the polyadic $\pi$-calculus into
the monadic $\pi$-calculus.
\end{enumerate}
%1, 2, 1; 4 marks.

\item Propose an alternative implementation of the incrementor and auxiliary
processes that is safe under concurrent execution. You may use the polyadic
$\pi$-calculus. Briefly argue why your implementation is safe (a reduction
is not required).
% 3 marks.
\end{enumerate}
[11 marks]

\item Consider the problem of actually implementing the actor model in a distributed
system. We introduce several concepts and ideas from Microsoft's \textit{Orleans} 
framework \cite{orleans}. So...

\begin{enumerate}
\item Show that bar.
\end{enumerate}

\end{enumerate}



\newpage
\begin{thebibliography}{9}
\bibitem{endsleigh}
``77\% of students now work to fund studies." Accessed 2 January 2017. $<$\url{https://www.endsleigh.co.uk/press-releases/10-august-2015/}$>$

\bibitem{orleans}
Bernstein, Philip A., et al. ``Orleans: Distributed Virtual Actors for Programmability and Scalability."

\end{thebibliography}

\end{document}